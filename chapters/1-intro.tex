% SPDX-FileCopyrightText: 2023 Oscar Bender-Stone <oscarbenderstone@gmail.com>
% SPDX-License-Identifier: CC-BY-4.0
% 1-intro.tex - Introduction for Welkin Book

% TODO Add proper bibtex references

\chapter{Introduction}
\label{ch:intro}

% Starting point for Deleuze: https://plato.stanford.edu/entries/deleuze/ https://link.springer.com/chapter/10.1057/9780230280731_11

Startng Outline:
\begin{itemize}
  \item Describe the ways philosophers and mathematicians have approached organizing the world around them.
        \begin{itemize}
          \item Look at Cantor, Dedekind, Frege, Russell, Goedel, and up to the present.
          \item Look at modern philosophers
        \end{itemize}
  \item Reflect on two major shortcomings of these approaches: having little metaphysical explanation about \textit{how} things can be combined, and a lack of generality for several concepts
        \begin{itemize}
          \item No explanation for what \textit{is} a set fundamentally, or, more importantly, how two entities can be combined into one.
                \begin{itemize}
                  \item Include formal and informal definitions
                  \item Recognize the prior \textit{tool} used to start math in the first place (again, the ability to combine objects or ideas)
                  \item Explain that, while category theory may make these constructions more precise, it \textit{still relies on predetermined tools}.
                        These tools are physiological in nature, but they are \textit{ultimately metaphysical.}
                \end{itemize}
          \item Lack of generality for: structures, the real numbers, logic (particularly paraconsistent logics). Specifically, mention:
                \begin{itemize}
                  \item Cryptomorphisms and Voldemort's Theorem. This is not part of official literature, but has been given in an extensive document. It serves an important philosophcial point (which will be rexammined in Chapter 2)
                  \item Mention the number of models for the reals, and how it still has not been decided what the official model is. Mention constructive concerns (overarrching with generality). (Cite Lesnik's paper as a step in the right direction, but critique the lack of generality)
                  \item Mention the role of bifrucation, and that there is opportunity to expand upon it more (into a generalized fuzzy logic)
                \end{itemize}
  \end{itemize}
  \item Explain the essential goals of Welkin, built upon CFLT
  \item Describe target audience
  \begin{itemize}
    \item This document is geared towards philosophically or mathematically inclined persons. The official version is in English, but, in time, translations will be added.
    \item In a different document (separate from this repo), there will be a more accessible version as a part of my program on ``humanistic logic''.
  \end{itemize}
\end{itemize}
