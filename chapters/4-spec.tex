% SPDX-FileCopyrightText: 2023 Oscar Bender-Stone <oscarbenderstone@gmail.com>
% SPDX-License-Identifier: CC-BY-4.0
% 4-spec.tex - Specifices the Welkin language and provides a theoretical framework
% for general implementations
\chapter{The Welkin Standard}

\section{The Welkin Language}

\begin{itemize}
  \item Determine a suitable BNF for Welkin, which can be parsed with LALR (or otherwise a more efficient parser)
  \begin{itemize}
    \item Key goal: make Welkin's syntax fully decidable and efficient to parse. An important component of CFLT called the Semantics Lifting Lemma (TBD) essentially says we can embed a complex syntax into a semantics. (This proof will hopefully be constructive and work for any random syntax, no matter how crazy it might be). In other words, using an efficient parser does NOT limit how expressive Welkin is.
  \end{itemize}
  \item Following CFLT, explain a suitable semantics for Welkin.
        \begin{itemize}
          \item We need to determine how to implement all of the axioms.
          \item We also need to use a suitable proof system (e.g., Hilbert, Gentzen, etc.). Maybe that could be decided in CFLT?
        \end{itemize}
\end{itemize}

\section{Core Algorithms}

\section{General Application Behavior}

\label{ch:spec}
