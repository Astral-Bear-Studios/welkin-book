% SPDX-FileCopyrightText: 2023 Oscar Bender-Stone <oscarbenderstone@gmail.com>
% SPDX-License-Identifier: CC-BY-4.0
% 2-background.tex - Covers problems related to organization, prexisting tools, and the overal motivation
% of Welkin
\chapter{Background}
\label{ch:background}
We address several issues that arise from organization, specifically within mathematics and computer science. We stress that they are not restricted to these fields and are widely applicable to all others.

\section*{Abstractions}

\section*{Fragmentation}

\section*{Limitations in Extensions} %TODO: rename

\section*{Complexity} Cite sources of lines of code, mathematical documents, etc.

Cite: ``How do Committees Invent'', Conway: ``https://www.melconway.com/Home/pdf/committees.pdf'' (While the sources in this paper have NOT been formally justified, we are addressing their concerns.)




% TODO: put the following text in a separate repositoriy
% Possible beginning quote (at least for Western philosophers). ``The Fold: Leibniz and the Barouqe''
% \begin{displayquote}
% the two instances . . . have no windows, . . . for Leibniz, [this! is because the monad’s being-for the world is subject to a condition of closure, all compossible monads including a single and same world. For Whitehead, on the contrary, a condition of opening causes all prehension to be already the prehension of another prehension. . . . Prehension is naturally open, open to the world, without having to pass through a window.
% \end{displayquote}

% \section{Western and Eastern Thought}

% \begin{itemize}
%   \item Contrast the distinct philosophies pervasive in Western and Eastern philosophy. Some key points to emphasize:
%         \begin{itemize}
%           \item Mechanistic vs spiritual leaning
%           \item Classical logic vs Different Logics (particularly involving contradictions)
%           \item Brief mention in other philosophies (possibly ethics or other non-epistemology related fields?)
%         \end{itemize}
% \end{itemize}

% \section{Sets: Georg Cantor and Others}
% \begin{itemize}
%   \item Cover most of Cantor's writings, including ``Beiträge zur Begrundung der transfiniten Megenlehre''.
%         \begin{itemize}

%           \item Touch upon other thinkers that built upon Cantor, including Frege, Dedekind, etc. Dedekind particularly has a section on systems that he later uses to define naturals (and is extremely relevant to Cantor's definition).

%           \item Cover the core issues behind Cantor's original definition, including pyschological components. Use modern crtiqiues in current literature. (See references in: ``New Essays On Peirce's Mathematical Philosophy'')

%           \item Poin

%                 \end{itemize}
%   \item Explain the overreaching implications of set theory, as well as its materialistic issues. Briefly explore the possible resolutions to this (e.g., type theory), and demonstrate that they do not sufficiently resolve the materialistic roots.
%   \item Explore the issue of defining an object, claiming a more general theory is possible. Argue that Cantor and others have used foci

%   \item Touch upon the formalist response to set theory, and rebuttal against the claim that formalism alone can be used for mathematical thought.
%   \item Helpful: look at the references in the metamath book
% \end{itemize}


% % TODO Decide whether to make separate chapters on foci and continua. I believe it would be benefitical to do so; I already split my first section on foci into two parts, and continua may deserve the same treatment. As of now, I will keep track of their relative lengths.

% \section{The Role of Foci}
% \begin{itemize}
%   \item Explore the issue of defining an object, claiming a more general theory is possible. Argue that Cantor and others have used foci (in a specialized form, collections)
%   \item Justify the existence of different paradoxes the applicability of non-classical logics (particularly paraconsistent logic)
%   \item Explain the shortcomings of foci alone in explaining both new phenomena and generality
% \end{itemize}

% \section{Early Attempts at Continua: Cantor, Dedekind, and Others}
% \begin{itemize}
%   \item Revisit Cantor, particularly with Cantor's Theorem. Explore the different models of the real numbers.
%   \item Referencing foci, explain how these models do not completley generalize the reals. Eve mention how category theory does not resolve this either (in both a mathematical and philosophical sense).
%   \item Analyze Zeno's paradoxes and Aristotle's original definition of a continuum. Explain set-theoretic continua do not constitute full fleged continua (and are missing key metaphysical properties)
% \end{itemize}

% \section{Continua: Charles Peirce}
% \begin{itemize}
%   \item Discuss Peirce's role in mathematics, logic, and ontology.
%   \item Briefly mention Pragmatism and Peirce's search for its proof.
%   \item Explain what Perice thought about Cantor's theory, and introduce his concept of multitudes
%   \begin{itemize}
%     \item Elaborate on his initial readings of Cantor's theorem and definition of sets
%     \item Explore Peirce's shortcomings in his definition of collection
%     \item Mention some of his misconceptions, e.g., his presumption that the Continuum Hypothesis must hold
%   \end{itemize}
%   \item Informally justify the mathematical value of Peirce's writings, arguing that more concrete axioms need to be developed from his ideas. In particular, explain why Perice was closer than Cantor to getting to a definitin of a bona fide continuum.
% \end{itemize}

% \section{Folds: Deleuze}

% Possible quote to include about Deleuze, ``The Fold: Leibniz and the Baroque'', page 81 (maybe this is better suited for the introduciton?)
% \begin{displayquote}
% For Leibniz... bifrucations and divergencies of series are genuine borders between incompossible worlds, such that the monads that exist wholly include the compossible world that moves into existence. For Whitehead (and for many modern philosophers)... bifrucations, divergences, incompossibilites, and discord belong to the same motley world that can no longer be included in expressive units, but only made or undone  according to prehensive units and variable configurations or changing captures. In a same chaotic world divergent series are endlessly tracing bifurating paths. It is a ``chaosmos'' of the  type found in Joyce, but also in Maurice Leblanc, Borges, or Gomrowicz.
% \end{displayquote}

% \begin{itemize}
%   \item Describe Deleuze's work in epistemology and relevance throughout modern philosophy.
%   \item Connect Deleuze to some concepts in Eastern thought, particularly tying his idea of folds with origami. Also mention how he challenged Leibniz's metaphysics.
%   \item Explain how Deleuze's concept of a fold resolves Peirce's previous inquries into collections.
% \end{itemize}
