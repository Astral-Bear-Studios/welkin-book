% SPDX-FileCopyrightText: 2023 Oscar Bender-Stone <oscarbenderstone@gmail.com>
% SPDX-License-Identifier: CC-BY-4.0
% welkin-book.tex - Sets up the main Welkin Book settings, packages, and chapters
\documentclass[leqno]{book}
\usepackage[subpreambles=true]{standalone}

\usepackage{titling}
\usepackage{outlines}
\usepackage{csquotes}
\usepackage{hyperref}
\usepackage{simplebnf}
\usepackage{amsfonts}
\usepackage{amsthm}
% Needed to avoid underscores merging into characters; see https://tex.stackexchange.com/questions/48632/underscores-in-words-text
\usepackage[T1]{fontenc}

\theoremstyle{definition}
\newtheorem{definition}{Definition}[section]
\newtheorem{theorem}{Theorem}[section]
\newtheorem{lemma}[theorem]{Lemma}


\title{Welkin \\
  An Information Language}
\author{Oscar Bender-Stone}
\date{October 2023}


\begin{document}
\maketitle

%\frontmatter
%\tableofcontents

\chapter*{Preface}

To be written...


Chapters 1 and 2 provide the philosophical basis for Welkin. Chapter 3 contains the Welkin Standard, a formal specification describing high level, algorithmic properties in compliant Welkin interpreters. Chapter 4 explores several ways to use Welkin. Chapter 5 concludes with the future of digitally storing mathematical documents, synthesized programs, and the overall organization of ideas.


\mainmatter

% SPDX-FileCopyrightText: 2023 Oscar Bender-Stone <oscarbenderstone@gmail.com>
% SPDX-License-Identifier: MIT
% 1-intro.tex - Introduction for Welkin Book

% TODO Add proper bibtex references

\chapter{Introduction}
\label{ch:intro}

% Starting point for Deleuze: https://plato.stanford.edu/entries/deleuze/ https://link.springer.com/chapter/10.1057/9780230280731_11

Startng Outline:
\begin{itemize}
  \item Describe the ways philosophers and mathematicians have approached organizing the world around them.
        \begin{itemize}
          \item Look at Cantor, Dedekind, Frege, Russell, Goedel, and up to the present.
          \item Look at modern philosophers
        \end{itemize}
  \item Reflect on two major shortcomings of these approaches: having little metaphysical explanation about \textit{how} things can be combined, and a lack of generality for several concepts
        \begin{itemize}
          \item No explanation for what \textit{is} a set fundamentally, or, more importantly, how two entities can be combined into one.
                \begin{itemize}
                  \item Include formal and informal definitions
                  \item Recognize the prior \textit{tool} used to start math in the first place (again, the ability to combine objects or ideas)
                  \item Explain that, while category theory may make these constructions more precise, it \textit{still relies on predetermined tools}.
                        These tools are physiological in nature, but they are \textit{ultimately metaphysical.}
                \end{itemize}
          \item Lack of generality for: structures, the real numbers, logic (particularly paraconsistent logics). Specifically, mention:
                \begin{itemize}
                  \item Cryptomorphisms and Voldemort's Theorem. This is not part of official literature, but has been given in an extensive document. It serves an important philosophcial point (which will be rexammined in Chapter 2)
                  \item Mention the number of models for the reals, and how it still has not been decided what the official model is. Mention constructive concerns (overarrching with generality). (Cite Lesnik's paper as a step in the right direction, but critique the lack of generality)
                  \item Mention the role of bifrucation, and that there is opportunity to expand upon it more (into a generalized fuzzy logic)
                \end{itemize}
  \end{itemize}
  \item Explain the essential goals of Welkin, built upon CFLT
  \item Describe target audience
  \begin{itemize}
    \item This document is geared towards philosophically or mathematically inclined persons. The official version is in English, but, in time, translations will be added.
    \item In a different document (separate from this repo), there will be a more accessible version as a part of my program on ``humanistic logic''.
  \end{itemize}
\end{itemize}

% SPDX-FileCopyrightText: 2023 Oscar Bender-Stone <oscarbenderstone@gmail.com>
% SPDX-License-Identifier: CC-BY-4.0
% 2-background.tex - Covers the philosophical and mathematical background
% leading up to Continuum Foci Logic and Theory
\chapter{Background}
\label{ch:background}

Possible beginning quote (at least for Western philosophers). ``The Fold: Leibniz and the Barouqe''
\begin{displayquote}
the two instances . . . have no windows, . . . for Leibniz, [this! is because the monad’s being-for the world is subject to a condition of closure, all compossible monads including a single and same world. For Whitehead, on the contrary, a condition of opening causes all prehension to be already the prehension of another prehension. . . . Prehension is naturally open, open to the world, without having to pass through a window.
\end{displayquote}

\section{Western and Eastern Thought}

\begin{itemize}
  \item Contrast the distinct philosophies pervasive in Western and Eastern philosophy. Some key points to emphasize:
        \begin{itemize}
          \item Mechanistic vs spiritual leaning
          \item Classical logic vs Different Logics (particularly involving contradictions)
          \item Brief mention in other philosophies (possibly ethics or other non-epistemology related fields?)
        \end{itemize}
\end{itemize}

\section{Sets: Georg Cantor and Others}
\begin{itemize}
  \item Cover most of Cantor's writings, including ``Beiträge zur Begrundung der transfiniten Megenlehre''.
  \item Touch upon other thinkers that built upon Cantor, including Frege, Dedekind, etc.
  \item Cover the core issues behind Cantor's original definition, including pyschological components.
  \item Explain the overreaching implications of set theory, as well as its materialistic issues. Briefly explore the possible resolutions to this (e.g., type theory), and demonstrate that they do not sufficiently resolve the materialistic roots.
  \item Explore the issue of defining an object, claiming a more general theory is possible. Argue that Cantor and others have used foci
\end{itemize}

% TODO Decide whether to make separate chapters on foci and continua. I believe it would be benefitical to do so; I already split my first section on foci into two parts, and continua may deserve the same treatment. As of now, I will keep track of their relative lengths.

\section{The Role of Foci}
\begin{itemize}
  \item Explore the issue of defining an object, claiming a more general theory is possible. Argue that Cantor and others have used foci (in a specialized form, collections)
  \item Justify the existence of different paradoxes the applicability of non-classical logics (particularly paraconsistent logic)
  \item Explain the shortcomings of foci alone in explaining both new phenomena and generality
\end{itemize}

\section{Early Attempts at Continua: Cantor, Dedekind, and Others}
\begin{itemize}
  \item Revisit Cantor, particularly with Cantor's Theorem. Explore the different models of the real numbers.
  \item Referencing foci, explain how these models do not completley generalize the reals. Eve mention how category theory does not resolve this either (in both a mathematical and philosophical sense).
  \item Analyze Zeno's paradoxes and Aristotle's original definition of a continuum. Explain set-theoretic continua do not constitute full fleged continua (and are missing key metaphysical properties)
\end{itemize}

\section{Continua: Charles Peirce}
\begin{itemize}
  \item Discuss Peirce's role in mathematics, logic, and ontology.
  \item Briefly mention Pragmatism and Peirce's search for its proof.
  \item Explain what Perice thought about Cantor's theory, and introduce his concept of multitudes
  \begin{itemize}
    \item Elaborate on his initial readings of Cantor's theorem and definition of sets
    \item Explore Peirce's shortcomings in his definition of collection
    \item Mention some of his misconceptions, e.g., his presumption that the Continuum Hypothesis must hold
  \end{itemize}
  \item Informally justify the mathematical value of Peirce's writings, arguing that more concrete axioms need to be developed from his ideas. In particular, explain why Perice was closer than Cantor to getting to a definitin of a bona fide continuum.
\end{itemize}

\section{Folds: Deleuze}

Possible quote to include about Deleuze, ``The Fold: Leibniz and the Baroque'', page 81 (maybe this is better suited for the introduciton?)
\begin{displayquote}
For Leibniz... bifrucations and divergencies of series are genuine borders between incompossible worlds, such that the monads that exist wholly include the compossible world that moves into existence. For Whitehead (and for many modern philosophers)... bifrucations, divergences, incompossibilites, and discord belong to the same motley world that can no longer be included in expressive units, but only made or undone  according to prehensive units and variable configurations or changing captures. In a same chaotic world divergent series are endlessly tracing bifurating paths. It is a ``chaosmos'' of the  type found in Joyce, but also in Maurice Leblanc, Borges, or Gomrowicz.
\end{displayquote}

\begin{itemize}
  \item Describe Deleuze's work in epistemology and relevance throughout modern philosohpy.
  \item Connect Deleuze to some concepts in Eastern thought, particularly tying his idea of folds with origami. Also mention how he challenged Leibniz's metaphysics.
  \item Explain how Deleuze's concept of a fold resolves Peirce's previous inquries into collections.
\end{itemize}

% SPDX-FileCopyrightText: 2023 Oscar Bender-Stone <oscarbenderstone@gmail.com>
% SPDX-License-Identifier: CC-BY-4.0
% 4-spec.tex - Specifices the Welkin language and provides a theoretical framework
% for general implementations

% TODO: create a version for welkin-standard.tex that directly includes the copyright notice
% Henceforth, compliant Welkin implementations will be collectively referred to as ``Welkin.'' A formal definition of

We now provide the full specification of the Welkin language. Everything beyond this paragraph is included in the official standard. Note that the entire standard, including its references, is self contained.


This standard requires a cursory background in discrete mathematics, parsing, and Backus-Naur Form (BNF). A reading of [] and [] suffices to understand this document.

\section{Preliminaries}

% Helpful macros for terms for character encoding in math mode
\newcommand*{\chars}{\mathrm{CHAR}}
\newcommand*{\numbers}{\mathrm{NUMBER}}
\newcommand*{\whitespaces}{\mathrm{WHITE\_SPACES}}
\newcommand*{\reserved}{\mathrm{RESERVED}}
\newcommand*{\strings}{\mathrm{STRING}}
\newcommand*{\term}{\mathrm{term}}
\newcommand*{\terms}{\mathrm{terms}}
\newcommand*{\delimiters}{\mathrm{DELIMITERS}}
\newcommand*{\encoding}{\mathcal{E}}
\newcommand*{\decoding}{\mathcal{D}}

\newcommand*{\scope}{\textrm{scope}}

\SetBNFConfig{
  relation = {::=|=>},
  relation-sym-map = {
    {::=} = {=},
    {=>} = {->}
  }
}

\subsection{Character Encodings}
% TODO: generalize to byte encodings and numbers.
In a formalist fashion, we define text, character encodings, and character decodings as generalized notions. The discussion here may be carried out in terms of bytes and with specific data formats, but these concepts are beyond the scope of this standard.

Let Char be an arbitrary, finite set. An \textbf{encoding} is an injective mapping $\encoding : \mathbb{N} \to \textrm{Char}.$ The associated \textbf{decoding} is the left-inverse $\decoding: \mathrm{Char} \to \mathbb{N}$ of $\encoding.$ There is a natural extension $\decoding^{*}: \textrm{Char}^{*} \to \mathbb{N}^{*}$
that maps sequences in Char pointwise to sequences in $\mathbb{N}^{*}.$

Character encodings may be given as finite tables, matching natural numbers with characters. Several major encodings are formally defined in the following sources.
\begin{itemize}
	\item ASCII []
	\item UTF-8 []
	\item UTF-16 []
\end{itemize}

We denote $\chars = \decoding(\textrm{Char})$ and CHAR* as the
Kleene-closure of CHAR, whose elements are called \textbf{words.}\footnote{Traditionally, the Klenne-closure of a set $A$ is denoted by $A^{*}.$ However, to ensure our BNF
  can be written in pure ASCII, we append $*$ without a superscript on $\chars$ and its subsets.}
% TODO: put this into table format. We need an easy way to reference any of these sets and notes about them
% Defining each subimage linearly is not working that well.
We list several secondary notions in Table ?.? All of these sets, except STRING, are arbitrary.
\begin{center}
  \bgroup
  \def\arraystretch{2.0}
\begin{tabular}{| c | c | c |}
  \hline
  \textbf{Set} & \textbf{Definition} & \textbf{Notation} \\
  \hline
  NUMBERS & Subset of $\chars*$ & $r$ \\
  \hline
  WHITE\_SPACES & Subset of $\chars*$ & $ws$ \\
  \hline
  DELIMITERS & Subset of $(\chars*)^{2}$ & $(d_{1}, d_{2})$ \\
  \hline
  STRING & \makecell{$s = d_{1}ud_{2},$ \\ $u \in \chars*, u \neq d_{1}, d_{2}.$ \\ $u$ is the \textbf{contents of} $s$ } & \makecell{$s,$ with \\ contents $\hat{s}$} \\
  \hline
\end{tabular}
\egroup
\end{center}

Strings may include their delimiters by escaping them, i.e., using a distinct prefix or suffix DELIMITER\_ESCAPE.

Every standard Welkin grammar is written in ASCII, but the interpreter may support additional encodings. (See Section ?.?).

We implicitly assume that $\chars*$ does not conflict with literals defined in a given standard grammar. In terms of the recommended LALR parser, this means that literals are matched first, not identifiers. However, these characters may be used by creating a custom grammar (see Section ?).


\subsection{BNF Variant}
Our variant of BNF uses the notation in Table ?.?.
\begin{center}
  \begin{tabular}{ | c | p{2cm} | p{6cm} | }
  \hline
  \textbf{Concept} & \textbf{Notation} & \textbf{Example} \\
  \hline
	Rule Assignment & $=$ & \begin{bnf} term ::= atom\end{bnf}\\
  \hline
  Empty Word & $\varepsilon := \emptyset$ & \begin{bnf} term ::= $\varepsilon$ \end{bnf} \\
  \hline
  \makecell{Concatenation \\ (No white spaces \\ inbetween rules)} & \makecell{Separate with \\ .}&  \begin{bnf} operation ::= operator.'('.string.')'\end{bnf} \\
  \hline
  \makecell{Concatenation \\ (White spaces \\ allowed)} & \makecell{Separate with \\ white space} & \begin{bnf} data ::= date name\end{bnf} \\
    \hline
    Literals & `word' & \begin{bnf} boolean ::= `true' || `false' \end{bnf} \\
    \hline
  Choice Names & terms $\to$ rule & \makecell{\begin{bnf} boolean ::= `true' $\to$ true || `false' $\to$ false \end{bnf}, \\ equivalent to \\ \begin{bnf}  boolean ::= true // false ;; true ::= `true' ;; false ::= `false' \end{bnf}} \\
    \hline
\end{tabular}
\end{center}
Each BNF has an associated subset $\reserved \subseteq \chars*$ for any literals that appear in the grammar. We will explicty state these for standard Welkin grammars in the next section.
% TODO: convert this into a table for easy access
% \begin{itemize}
%   \item $::=$ denotes rule assignment,
%   \item \texttt{term ::= $S \subseteq \chars^{*}$} means \texttt{term $\in S$},
%   \item $|$ denotes alternation,
% 	\item In a given choice, an arrow $\to$ denotes a new rule name. For example, the rule
% 		\begin{bnfgrammar}
% 			term ::= `A' $\to$ A | `B' $\to$ B
% 		\end{bnfgrammar}
% 		is equivalent to
% 		\begin{bnfgrammar}
% 			term ::= A || B ;;
% 			A ::= `A' ;;
% 			B ::= `B'
% 		\end{bnfgrammar}
% 	\item \texttt{term*} means 0 or more instances and is shorthand for
% 	\begin{center}
% 		\texttt{terms ::= term terms | term | $\varepsilon$},
% 	\end{center}
% 	\item \texttt{term+} means 1 or most one instances and is shorthand for
% 	\begin{center}
% 		\texttt{terms ::= term terms | term},
% 	\end{center}
%   \item Elements of CHAR* are included in quotes. To avoid confusion, literal quotes are denoted with ['].
% \end{itemize}
\section{The Welkin Language}

There are three fundamental variants of Welkin that define the foundation for the language:
\begin{itemize}
	\item Base Welkin, mirroring the key properties of the core data structure.
	\item Attribute Welkin, extending Base Welkin with attributes. Attributes are a limited type of directive that can customize how the interpreter accepts or presents data.
	\item Binder Welkin, enabling arbitrary evaluation of Welkin files and access to the user's operating system. This is equivalent to Attribute Welkin with two new directives: \texttt{@eval} and \texttt{@exec}. \end{itemize}
Each of these variants can be parsed with LALR parsers and fundamentally have the same semantics. However, in Binder Welkin, \texttt{@eval} makes the interpreter Turing complete (see Section ?.?), and using \texttt{@exec} can significantly impact the user's system. For this reason, Binder Welkin is a separate, optional component, as detailed in the Section ?.?
% \begin{itemize}
% 	% \item First Define character encodings in general. Helpful reference: \url{https://www.w3.org/International/questions/qa-what-is-encoding}
% 	%       \begin{itemize}
% 	% 	      \item For wide spread use, there should be different character encodings used for \textit{direct comparison} with Welkin files. Ultimately, every Welkin file will be converted into a standard binary (or possibly text) file to store the object
% 	%       \end{itemize}
% 	\item Determine a suitable BNF for Welkin, which can be parsed with LALR (or otherwise a more efficient parser)
% 	      \begin{itemize}
% 		      \item Key goal: make Welkin's syntax fully decidable and efficient to parse. An important component of CFLT called the Semantics Lifting Lemma (TBD) essentially says we can embed a complex syntax into a semantics. (This proof will hopefully be constructive and work for any random syntax, no matter how crazy it might be). In other words, using an efficient parser does NOT limit how expressive Welkin is.
% 		      \item Presuming the result above, there will be two variants: the finite (regex) and full versions.
% 		            \begin{itemize}
% 			            \item The finite, or regex, version is purely for regex-definable files.
% 			            \item The full version will be LALR parsed, as it is generally a standard for programming languages. Not only is it efficient (both in speed and memory), but any grammar written in LALR is unambiguous (reference needed!).
% 			            \item Now that the idea of these two versions is solidifed, we need some common terms. Most of these should come from graphviz, but also in other note taking formats.
% 		            \end{itemize}
% 		      \item The standard format should read just like an ordinary programming language. It may be akin to graphviz, but it should prioritze on the contents of each node and edge, not necessarially how they are rendered. (A better thought would be to put rendering information in a standard \textit{library}, which could then be minimized when browsing through a Welkin file/project.)
% 		            \begin{itemize}
% 			            \item Welkin essentially needs the key elements from set theory: conjunction, disjunction, negation, implication, etc. We can use corresponding symbols for these: $\&\&, ||, \neg, \rightarrow$. In \textit{customizable files}, these symbols can be overloaded and added upon.
% 			            \item Key goal: make this FULLY compatible with dot. (In fact, for a prototype, we can work with dot directly, but we should make it helpful for our needs).
% 									\item Another important point: we want to say that graph ALWAYS refers to a metagraph (to avoid redundancy)		            \end{itemize}
% 	      \end{itemize}
% 	\item Following CFLT, explain a suitable semantics for Welkin.
% 	      \begin{itemize}
% 		      \item We need to determine how to implement all of the axioms.
% 		      \item We also need to use a suitable proof system (e.g., Hilbert, Gentzen, etc.). Maybe that could be decided in CFLT?
% 	      \end{itemize}
% \end{itemize}
% TODO: figure out how to handle references to self. Is a separate keyword 'self' needed?
%\renewcommand{\syntleft}{\normalfont\bfseries}
%\renewcommand{\syntright}{}

% TODO: directly convert this into other grammars, such as lark.
% There should be some consistent procedure to ensure that the BNF here
% is the same as those found in any implementation
% Interesting idea: when there is a node A that should connect to other nodes B_1, B_2, ..., B_n,
% we require that the latter nodes be wrapped in their own graph. That way, we can stay consistent with
% GraphViz notation (for possible compatiblity reasons), but at the same time, we can keep track of the out-neighbors (out going
%\renewcommand{\bnfexpr}{\textbf}
\subsubsection*{Syntax}
Set $\reserved = \{\texttt{\{, \}, ., -, ->, <-}, @\}.$ Each grammar is provided in Table 3.?.
% TODO; figure out rule for when NUMBERS is empty
% TODO: make different RESERVED key words for Base and Attribute Welkin. Binder Welkin is fairly straightforward
% (We actually only need to look at prefixes; any attribute names WILL be parsed first. Welkin will assume
% the user wanted the built-in directives. A different name for those should be used)
% TODO: It may be better to type up Welkin in regular font; it looks readable and could be done well in math mode.

% \SetBNFConfig{

%   relation-sym-map = {
%     {::=} = {\ensuremath{=}},
%   }
%   }
%     TODO: Remove Notes column
% TODO: add -> back into the grammar; using -> directly conflicts with simplebnf
\begin{center}
    \begin{tabular}{| p{1.5cm} | p{9.5cm} |}%{1\textwidth}{| l | X |}
    \hline
    Variant & Grammar \\\hline % & Notes \\
      \makecell{Base \\ Welkin} &
  \begin{bnf}
  terms ::= term*;;
  graph ::= unit? `\{' terms `\}' ;;
  connections ::= term (connector term)+ ;;
	connector ::=
   | `-' term `-' $\to$ edge
   | `-' term `>' $\to$ left\_arrow
	 | `<-' term `-' $\to$ right\_arrow ;;
  member ::= unit? (`.'.(ident // string)? // `\#'.num )+ ;;
	unit ::= ident // string // num ;;
  ident ::= CHAR* ;;
	string ::= STRING ;;
	num ::= NUMBER
\end{bnf} \\ %& If $\numbers = \emptyset,$ any instance of \texttt{num} should be removed from the parser. \\
   \hline
      Attribute Welkin &
      \begin{bnf}
  statement ::= (directive // construct // term)* ;;
  directive ::= `@'.attribute ;;
  attribute ::= `import' tuple $\to$ import
  | test ;;
  construct ::= operator // tuple // list ;;
  operator ::= ;;
  tuple ::= ;;
  list ::=
 \end{bnf} \\
                      %& This is limited to the CLI and GUI (TBD). \\
   \hline
   Binder Welkin & \begin{bnf}
     directives ::= attributes // binders ;;
     binders ::= `eval'.`(' unit `)' $\to$ eval
     | `exec'.`(' string `)' $\to$ exec
   \end{bnf} \\ %& Intended for use by programmers, mathematicians, and scientists.\\
    \hline
 \end{tabular}
\end{center}

% relation-sym-map = {
%       {::=} = {\ensuremath{=}}
%	}


	% term ::= graph || connections || member || unit ;;
	% graph ::= unit? `\{' terms `\}' ;;
	% connections ::= term (connector term)+ ;;
	% connector ::= `-' term `-' $\to$ edge
	% | `-' term `->' $\to$ left\_arrow
	% | `<-' term `-' $\to$ right\_arrow ;;
	% member ::= unit? (`.'(ident || string)? || `\#'num )+ ;;
	% unit ::= ident || string || num ;;
	% ident ::= CHAR* ;;
	% string ::= STRING ;;
	% num ::= NUMBER


Throughout this document, Welkin documents are formatted with the following convention: the ASCII sequence \texttt{->} is rendered as $\to$ (A graphical user interface may support this rendering via glyphs). % TODO: put special renderings in a table
  % TODO: figure out suitable grammar composition notation
% Finally, Binder Welkin is given by the BNF in Attribute Welkin composed by two new directives. In BNF, these are appended to the directives rule:
% \begin{itemize}
% 	\item \texttt{},
% 	\item \texttt{}.
% \end{itemize}
% We explain the semantics for these directives in Section ?.?

% TODO: recognize, in cflt, that the above welkin file is in fact a context free grammar! The more important part, which we need to still define, is the semantics, which will have its full strength with the full grammar. (Maybe we should change that option to be semantics instead?)

\subsubsection*{Semantics}
We break down our semantics first by terms. Directives are handled separately in the next section.
\begin{definition}
Equality of terms.
\begin{itemize}
    % TODO: decide if ''A`` is the same as A
  \item \textbf{Basis.} Two units are equal if they are the same kind and obey one of the following.
	\begin{itemize}
	  \item \texttt{ident} terms are equal if their corresponding characters are equal,
      \item \texttt{string} terms are equal if their corresponding contents are equal. Thus, \texttt{``A''} coincides with \texttt{'A'},
	  \item \texttt{num} terms are equal if they represent the same value. Thus, \texttt{1} coincides with \texttt{10E}.
	\end{itemize}
  \item \textbf{Recursion.}
        \begin{itemize}
        \item Two members are the same if they contain the same list of units. % TODO: include the case of relative imports
        \item Two connectors are equal if they are equal as terms. %TODO: rework to make this clearer
 		  \item Two connections are equal if they connect the same terms and have equal connectors.
		  \item Two graphs are equal if they contain the same terms.
	\end{itemize}
\end{itemize}
\end{definition}
% This definition should be unnecessary; this should be clear from the BNF.
% \begin{definition} (Membership) Let $t$ be a term and $G$ a graph. We say $t$ is a \textbf{member} of $G$ if $t$ appears as the contents of $G.$\end{definition}
% \end{definition}
A \textbf{scope} is recursively defined and intutively is a level of terms.
\begin{definition} (Scope)
 Let $t$ be a term.
  \begin{itemize}
	\item If $t$ is not contained in a graph, then $\scope(\texttt{term}) = 0,$
	\item If $G' \in G$ are both graphs and $\scope(G') = n,$ then for all $t \in G',$ $\scope(t) = \scope(G) + 1.$
\end{itemize}
\end{definition}
A \textbf{valid} base Welkin file consists satisfies a unique naming rule: in every scope, there are no name collisons. In particular, every graph must \textbf{only be defined once.} Note that, by the way equality was defined between two numbers,
  there can only be one representation of a given number in a scope. For example, using \texttt{1} and \texttt{10E-1} in the same scope would produce a name collison.
\\ We first form an Abstract Syntax Tree (AST), from which we form the final stored data in a \textbf{Welkin Information Graph.}
\begin{definition}
  Base Welkin is parsed into the following AST $\mathcal{A}.$
  \begin{itemize}
	\item Every term is a new subtree with its contents as children.
    \item Every graph is an ordered pair of its aliases and list of children.
    \item Every connection is an ordered pair:
		  \begin{itemize}
			\item Left arrows $u \xrightarrow{e} v$ correspond to a triple $(u, e, v);$
			\item Right arrows $u \xrightarrow{e} v$ correspond to the triple $(v, e, u);$
			\item Edges correspond to both a left and right arrow.
		  \end{itemize}
	\item Every unit is converted into its corresponding encoding via $\encoding^{*}.$
  \end{itemize}
 \end{definition}
 % TODO: talk about encoding of numbers in the structure itself. This probably a separate encoding from the one used to write the welkin file
 % TODO: determine best term for 3-uniform hypergraph
   % TODO: see if there is a nicer definition in the literature (at least for n-hypergraphs)
\begin{definition}
	A \textbf{Welkin Information Graph (WIG)} $\mathcal{G} = (G, L, \{t, s, c, i, l\})$ consists of
  \begin{itemize}
    \item a sequence $G = (G_{0}, G_{1}, \cdots, G_{n})$ of sets called \textbf{layers}, whose contents are \textbf{cells,}
    \item a set $L$ of \textbf{labels} or \textbf{aliases},
    \item a family of functions $s_{k}, t_{k}, c_{k}: G_{k+1} \to G_{k}$ called the \textbf{source, target,} and \textbf{connector maps,} respectively, and
    \item a family of injective functions $i_{k}, l_{k}: G_{k} \to G_{k+1}$ called the \textbf{embedding} and \textbf{labeling maps,}
   \end{itemize}
  that obey the following equations:
  \begin{itemize}
    \item $s_{k} \circ c_{k + 1} = s_{k} \circ s_{k+1} = s_{k} \circ t_{k+1},$
    \item $t_{k} \circ c_{k + 1} = t_{k} \circ s_{k+1} = t_{k} \circ t_{k+1},$
    \item $c_{k} \circ c_{k + 1} = c_{k} \circ s_{k+1} = c_{k} \circ t_{k+1},$
    \item $s_{k} \circ i_{k} = \textrm{id}_{G_{k}} = t_{k} \circ i_{k} = c_{k} \circ i_{k}.$
  \end{itemize}
  Graphically, this definition can be displayed as a diagram
\begin{equation*}
  \begin{tikzcd}
  G_0 \arrow[r,"i_{0}"'] &
  G_1 \arrow[r, "i_k"] \arrow[l, "s_0"', bend right] \arrow[l, "t_0", bend left] \arrow[l, "c_0"] &
  G_2 \arrow[r, "i_k"] \arrow[l, "s_k"] \arrow[l, "t_k"] \arrow[l, "c_k"] &
  \cdots \arrow[r, "i_k"] \arrow[l, "s_k"] \arrow[l, "t_k"] \arrow[l, "c_k"] &
  G_n &
\end{tikzcd}
\end{equation*}
where the following diagrams commute (i.e., their composites are equal)
\begin{equation*}
  \begin{tikzcd}
  G_0 \arrow[r,"i_{0}"] &
  G_1 \arrow[r, "i_k"] \arrow[l, "s_0"] \arrow[l, "t_0"] \arrow[l, "c_0"] &
  G_2 \arrow[l, "t_k"] \arrow[l, "c_k"] &
\end{tikzcd}
\end{equation*}

  We call each member of $G_{k}$ a $k$\textbf{-cell}.
\end{definition}
We will abuse notation and write $\mathcal{G}$ for $G$ and $\mathcal{G}_{k}$ for $G_{k}.$
% Cite: WIGs are a special form of reflexive n-graphs, or n-globular sets. We add the labeling function to
% store a record of labels from a previous Welkin file, and the last property is used to distinguish internal arcs from
% connectors
The first three equations ensure that two cells $I_{1} = A_{1} \xrightarrow{e_{1}} B, I_{2} = A_{2} \xrightarrow{e_{2}} B_{2}$ are equal precisely when $(A_{1}, e_{1}, B_{1}) = (A_{2}, e_{2}, B_{2})$. In particular, we have $i(A) = A \xrightarrow{A} A,$ so we may associate each element of $G_{k}$ with a triple in $G_{k+1}.$ These equations mean that a WIG is, under cryptomorphism, equivalent to a list of nested ternary relations. Moreover, notice that our definition does not require every vertex to have an alias, as opposed to a colored graph (in which all vertices are colored).
% TODO: is the claim below true? What we want to say is that some structure may not always be apparent in a Welkin file (particularly if a high level grammar is used). To find more structure, more often than not, a new welkin annotation file is needed.
% There are several examples demonstrating that, under a suitable transformation, a normalized WIG may contain new structures not found in a Welkin file. See Example ??.

We provide the following definition that we will return to later in Section ?.?
\begin{definition} Let $\mathcal{G}, \mathcal{H}$ be WIGs. A \textbf{morphism} $\eta: \mathcal{G} \to \mathcal{H}$ is a family of maps $\eta_{k}: G_{k} \to H_{k}$ such that each square
\begin{equation*}
\begin{tikzcd}
  G_{k} \arrow[r, "i_{k}"] &%  \arrow[d, "\eta_{k}"] &
  G_{k+1} \arrow[l, "s_{k}"] \arrow[l, "t_{k}"] \arrow[l, "c_{k}"] &
  H_{k} \arrow[r, "i'_{k}"] &
  H_{k+1} \arrow[l, "s'_{k}"] \arrow[l, "t'_{k}"] \arrow[l, "c'_{k}"] &
\end{tikzcd}
\end{equation*}
commutes. An \textbf{isomorphism} is an invertible morphism, and an \textbf{automorphism} is an isomorphism $\alpha: \mathcal{G} \to \mathcal{G}.$
\end{definition}

\begin{lemma}
  The conversion from ASTs to Welkin Information Graphs is valid and is given by $\mu: \mathcal{T} \to \mathcal{W}, \mu(\mathcal{A}) = ...$
\end{lemma}
As a major consequence, for a $k$-cell $A$, $\eta(A \{ a \}) = A \xRightarrow{a} A,$ so the contents of a unit can be associated with the set of $k+1$-cells on $A.$

The final output of parsing is a normalized WIG. We define Welkin Canonical Form in the following fashion.
\begin{definition}
A WIG is in \textbf{Welkin Canonical Form (WCF)} if ...
\end{definition}
Based on this form, we have chosen a unique way to represent Welkin files. In particular, there is a representation under WIG (generalized) homotopies. We prove that there is a polynomial (or exponential?) algorithm to convert any WIG into WNF.

\section{General Application Behavior}

Note that all apparent structures may be adjusted under cryptomorphism.
\subsubsection*{Directives}
Each directive relies on the following components.
\begin{itemize}
  \item Parser: takes in a Welkin file and generates an AST,
  \item Validator: ensures that the AST is valid, raising an error that directly points to a violation,
  \item From here, an AST may be processed by three different means:
		\begin{itemize}
		  \item Recorder: takes the AST, converts it into a WIG in WCF, serializes the data,
		  \item Attributor: %TODO: change name!
		  \item Binder:
		\end{itemize}
\end{itemize}
\begin{center}
  \begin{tabular}{| c | c | c |}
	Directive & Definition & Example \\
	\hline
	\texttt{import} & Concatenates the file & Example

  \end{tabular}
\end{center}




\subsection{Customization}
All Welkin files are infinitely customizable via the welkin config file, which is written in Attribute Welkin. Any attribute can be used, and other Welkin files can be imported. A base config file is required to customize a Welkin grammar. From there, configs can be arbitrarily nested to create and connect any desired (context-free) grammar, validator, and displayer.
t has the following format:
% TODO: replace itemzie with listing
% TODO: decide how to work with a folder (or folders) of config files
\begin{itemize}
	\item Encoding
				\begin{itemize}
					% TODO: list major ascii versions/varieties. Need an official reference for this!
					\item Options: ascii, utf-8, utf-16, other
					\item In the case of other: we need to specify how to define an encoding. (We need a light-weight API for implementations)
				\end{itemize}
	\item Grammar
				\begin{itemize}
					\item Strength: bounded (only finitely nested graphs with a given nesting limit, no recursion), no-self (arbitrary nesting limit, but no recursion), full (recursion allowed)
					\item Customized: use a builtin template or custom welkin file. These can be used to change any part of the grammar, including adding keywords, the symbols used, adding new symbols, etc. Essentially, this will be a way to built new grammars from the original specification; we will need a separate parser for this (i.e., a parser of BNF/Welkin accepted notation).
				\end{itemize}

	\item (Optional) Language
				\begin{itemize}
					\item Defaults to English. Can be written in the writer's desired language (as long as it has been configured in Encoding above)
				\end{itemize}
\end{itemize}

% In Welkin, we informally write the BNF above as follows:
% % TODO: explain unit notation (as it maybe clearer than the recursion below). In other words, mark arbitrary variables with the keyword unit
% % TODO: decide whether to introduce another arrow symbol for custom grammars.
% % While we have imposed few to no restrictions on custom grammars (besides being LALR), it may be the case
% % that multiple people want to use => for their own purpose. Is there a convenient way we can do this?
% \begin{quote}{\ttfamily \raggedright \noindent
% 	term -> \{ graph connection ident string\}\\
% 	graph -> \{\{ident \{\}\}->`\{'--term--`\}'\}\\
% 	connection -> \{term--connector--term\}\\
% 	connector -> \{edge arrow\}\\
% 	edge -> `--'\\
% 	arrow -> `->'\\
% 	ident -> CHAR*\\
% 	string -> ``'' CHAR* ``'' | `\`' CHAR* `\''
% }\end{quote}

\section{Core Algorithms}

\subsection{Graph Encoding}



\label{ch:spec}

% SPDX-FileCopyrightText: 2023 Oscar Bender-Stone <oscarbenderstone@gmail.com>
% SPDX-License-Identifier: CC-BY-4.0
% 6.conclusions.tex - conclusions on Welkin, including overarching implicatinos/future programs


\end{document}
